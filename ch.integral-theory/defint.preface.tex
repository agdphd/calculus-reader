\documentclass[../book/calcnotes.tex]{subfiles}

\begin{document}
\section*{Preface}
\label{sec:defint.preface}
We now turn our attention to a brand new problem.

\begin{motprob}
  Given a continuous function $f$ which is non-negative over an interval $I = \closedint{a, b}$, calculate the area of the region above the $x$-axis and below the curve $y = f(x)$ over the interval $I$.
\end{motprob}

\begin{motex}
  Consider the function $f$ whose graph is shown in \cref{fig:boundregex}.
  What is the area of the shaded region?

  \begin{smallfig}
    \centering
    \begin{asy}
      smallsize();

      real f(real x) {return 1.5+sin(x^2);}

      real start=0, end=3.5;

      real a=1, b=3;

      xaxis("$x$",start,end,Ticks(new real[] {a,b}), Arrow);
      yaxis("$y$",0, Arrow);

      funcplot(f, start, end);
      fillunder(f, a, b, box1);
    \end{asy}
    \caption{Region bounded by a function over $\closedint{1, 3}$}
    \label{fig:boundregex}
  \end{smallfig}
\end{motex}

This kind of problem has a long history, and many of the greatest mathematical thinkers of the last millenium have dedicated substantial energy to it.
Fortunately, it won't take \emph{us} a thousand years to solve!
Still, we do have our work cut out for us.
\end{document}

%%% Local Variables:
%%% mode: latex
%%% TeX-master: "../book/calcnotes.tex"
%%% End: