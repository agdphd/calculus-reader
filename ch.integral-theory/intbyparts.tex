\documentclass[../book/calcnotes.tex]{subfiles}

\begin{document}
\section{Integration by parts}
\label{sec:int.parts}
In \cref{sec:int.subs}, we developed an antidifferentiation procedure which allowed us to \enquote{undo} the Chain Rule of \cref{sec:chainrule}.
Can we do the same thing for the Product Rule of \cref{sec:productrule}?

As a first step, let's see what happens if we simply antidifferentiate both sides of \cref{eq:productrule}.
\begin{theorem}
  \label{thm:productrule.reversed}
  Suppose $f$ and $g$ are differentiable functions.
  Then the general antiderivative of $f' \pbrac{x} g \pbrac{x} + f \pbrac{x} g' \pbrac{x}$ is $f \pbrac{x} g \pbrac{x} + C$.
  In other words,
  \begin{equation}
    \label{eq:productrule.reversed}
    \Int{\pbrac[\big]{f' \pbrac{x} g \pbrac{x} + f \pbrac{x} g' \pbrac{x}}}{x} = f \pbrac{x} g \pbrac{x} + C.
  \end{equation}
\end{theorem}

Yikes!
When we developed the idea of integration by substitution, we were a bit spooked by the mess on the left-hand side of \cref{eq:subs.long}, but that wasn't nearly as nasty as the left-hand side of \cref{eq:productrule.reversed}.
It turned out, of course, that we could in fact use \cref{eq:subs.long} to calculate antiderivatives, since (somewhat unexpectedly) the form of the left-hand side was relatively easy to find.
It seems a bit too much to hope that the same will be true now; we're almost never going to see integrals that already look like the left-hand side of \cref{eq:productrule.reversed}.

So what can we do?
Rather than simply give up on integrating product functions altogether, let's try rearranging \cref{eq:productrule.reversed} a bit.
We're hoping to find a formula that lets us compute integrals of products, so, as a first step, we should break up the integral on the left-hand side into two separate integrals of product functions:
\begin{equation}
  \label{eq:int.byparts.wrongside}
  \Int{f' \pbrac{x} g \pbrac{x}}{x} + \Int{f \pbrac{x} g' \pbrac{x}}{x} = f \pbrac{x} g \pbrac{x} + C.
\end{equation}
Of course, this only helps us if we're trying to calculate \emph{two} integrals!
This seems even less useful than the original form.

Still, we've started down this road, so let's keep going.
If we move one of the integrals to the other side of \cref{eq:int.byparts.wrongside}, we'll at least have something that treats one integral at a time.
\begin{equation}
  \label{eq:int.byparts.prelim}
  \Int{f \pbrac{x} g' \pbrac{x}}{x} = f \pbrac{x} g \pbrac{x} - \Int{f' \pbrac{x} g \pbrac{x}}{x}.
\end{equation}
(The $+C$ term is no longer needed, since there are indefinite integrals on both sides.)

Is \emph{this} of any use?
Remarkably, it is!
It tells us that we can \emph{replace} a product integral with another one, at the cost of differentiating one term and antidifferentiating the other.
In many circumstances, this will actually improve our situation quite a bit.

\begin{example}
  \label{ex:parts.easy.funcnames}
  Find $\Int{x \cos x}{x}$.
\end{example}

\begin{soln}
  The integrand $x \cos x$ is clearly a product function, with factors $x$ and $\cos x$.
  Our situation would be substantially improved if we differentiated the $x$ term (since that would eliminate it!), and antidifferentiating the $\cos x$ term won't hurt anything (since that will simply replace it with $\sin x$).
  Thus, we take $f(x) = x$ and $g'(x) = \cos x$; then, by \cref{eq:int.byparts.prelim}, we have
  \begin{align*}
    \Int{x \cos x}{x} &= \Int{f \pbrac{x} g' \pbrac{x}}{x} \\
    &= f \pbrac{x} g \pbrac{x} - \Int{f' \pbrac{x} g \pbrac{x}}{x} \\
    &= x \sin x - \Int{1 \cdot \sin x}{x} \\
    &= x \sin x - \pbrac{- \cos x}\\
    &= x \sin x + \sin x + C.
  \end{align*}
\end{soln}

And there we have it!
This technique will allow us to integrate many product functions.
The key, of course, will be finding the right decomposition of the integrand into $f$ and $g'$---in particular, one for which $\Int{f'(x) g(x)}{x}$ is more manageable than $\Int{f(x) g'(x)}{x}$ was.

Just as with substitution, though, this technique is going to result in a lot of messy bookkeeping.
We'll use a similar notational trick to help us manage things.
When we were finding an integral $\Int{f' \pbrac*{g \pbrac{x}} g' \pbrac{x}}{x}$, we made the change of variables $u = g \pbrac{x}$ and let $\diff{u} = g' \pbrac{x} \, \diff{x}$.
Now, we'll need to work with four pieces: $f$, $f'$, $g$, and $g'$.
Conventionally, we'll let $u = f \pbrac{x}$, $\diff{u} = f' \pbrac{x} \, \diff x$, $v = g \pbrac{x}$, and $\diff{v} = g' \pbrac{x} \, \diff{x}$; then \cref{eq:int.byparts.prelim} can be rewritten very compactly.

\begin{theorem}[Integration by parts] %TODO: Explain "parts" in theorem name?
  \label{thm:int.byparts}
  Suppose $f$ and $g$ are differentiable functions.
  Let $u = f \pbrac{x}$, $\diff{u} = f' \pbrac{x} \, \diff{x}$, $v = g \pbrac{x}$, and $\diff{v} = g' \pbrac{x} \, \diff{x}$.
  Then
  \begin{equation}
    \label{eq:int.byparts}
    \Int{f \pbrac{x} g' \pbrac{x}}{x} = \boxed{ \Int{u}{v} = uv - \Int{v}{u} } = f \pbrac{x} g \pbrac{x} - \Int{g \pbrac{x} f' \pbrac{x}}{x}.
  \end{equation}
\end{theorem}

Now, let's give this theorem a test-drive with some examples.

\begin{example}
  \label{ex:parts.easy.uv}
  Find $\Int{x \cos x}{x}$ by parts.
\end{example}

\begin{soln}
  Let $u = x$ and $\diff{v} = \cos x \, \diff{x}$.
  (Then $\diff{u} = 1 \, \diff{x}$ and $v = \sin x$.)
  By \cref{eq:int.byparts}, we then have
  \begin{multline*}
    \Int{x \cos x}{x} = \Int{u}{v} = uv - \Int{v}{u} \\
    = x \sin x - \Int{1 \cdot \sin x}{x} = x \sin x + \cos x + C.
  \end{multline*}
  This agrees with the result of our work in \cref{ex:parts.easy.funcnames}, which is good news!
  (Of course, we really did exactly the same calculation only minor changes in the notation, so it's not exactly \emph{surprising} news.)
\end{soln}

\begin{example}
  \label{ex:byparts.tworounds}
  Find $\Int{x^{2} e^{x}}{x}$ by parts.
\end{example}

\begin{soln}
  Let $u = x^{2}$ and $\diff{v} = e^{x} \, \diff{x}$.
  (Then $\diff{u} = 2x \, \diff{x}$ and $v = e^{x}$.)
  By \cref{eq:int.byparts}, we then have\sidefootnote{At this point, we'll stop writing out all the intermediate $\Int{u}{v}$ terms to save space.}
  \begin{equation*}
    \Int{x^{2} e^{x}}{x} = x^{2} e^{x} - \Int{2x e^{x}}{x}.
  \end{equation*}
  This leaves us with a new integral to calculate.
  But this is another parts integral!
  Thus, we need to go around again.
  This time, we'll take $u_{2} = 2x$ and $\diff{v_{2}} = e^{x} \, \diff{x}$, so $\diff{u_{2}} = 2 \, \diff{x}$ and $v_{2} = e^{x}$.
  Applying \cref{eq:int.byparts} again, we have that
  \begin{equation*}
    \Int{x^{2} e^{x}}{x} = x^{2} e^{x} - \Int{2x e^{x}}{x} = x^{2} e^{x} - \pbrac*{2x e^{x} - \Int{2 e^{x}}{x}} = x^{2} e^{x} - 2x e^{x} + 2 e^{x} + C.
  \end{equation*}
\end{soln}

Applying integration by parts more than once is a minor nuisance computationally, but it's still good to know that the option is there.
This suggests a heuristic for parts decompositions: we should look for a factorization of our integrand into a $u$ which becomes simpler when differentiated and a $\diff{v}$ which doesn't become worse when antidifferentiated.
(Occasionally, we'll stumble on a $\diff{v}$ which actually gets simpler when antidifferentiated, but this is relatively rare.)

\subsection{Parts trickery}
\label{sec:int.byparts.tricks}
So far, we've only considered cases where the choices of $u$ and $\diff{v}$ are relatively obvious.
Just as with substitution, however, some of the most interesting applications of integration by parts come when we have to make sneaky choices for the new variables.
Let's close out the section with a few examples of this idea.

In \cref{ex:byparts.tworounds}, we saw a case where we had to use two \enquote{rounds} of integration by parts to break our integral down into something manageable.
Of course, we can construct examples that require any number of rounds you like---just try calculating $\Int{x^{7} e^{x}}{x}$ to see how this can happen.
But things can get even more interesting than this!

\begin{example}
  \label{ex:byparts.cyclic}
  Find $\Int{e^{x} \sin x}{x}$ using integration by parts.
\end{example}

\begin{soln}
  It looks like we should split this integrand up as a product of $e^{x}$ and $\sin x$---but which should be $u$ and which should be $\diff{v}$?
  Let's have a go with $u = \sin x$ and $\diff{v} = e^{x} \, \diff{x}$ (so $\diff{u} = \cos x \, \diff{x}$ and $v = e^{x}$).
  Applying \cref{eq:int.byparts} then gives us that
  \begin{equation}
    \label{eq:ex.byparts.cyclic.firstu}
    \Int{e^{x} \sin x}{x} = e^{x} \sin x - \Int{\cos x e^{x}}{x}. \tag{\dag}
  \end{equation}
  That doesn't look like much of an improvement!
  What if we'd tried $u = e^{x}$ and $\diff{v} = \sin x \diff{x}$ instead?
  Then we'd get
  \begin{equation*}
    \Int{e^{x} \sin x}{x} = -e^{x} \cos x - \Int{-\cos x e^{x}}{x}.
  \end{equation*}
  Either way, we don't seem to make any progress---the new integral isn't any better than the old integral!
  Before we give up hope, though, let's try the same technique we used in \cref{ex:byparts.tworounds}.
  Returning to \cref{eq:ex.byparts.cyclic.firstu}, let's make a second-round substitution, letting $u_{2} = \cos x$ and $\diff{v_{2}} = e^{x} \, \diff{x}$ in the integral on the right-hand side.
  Applying \cref{eq:int.byparts} a second time gives us that
  \begin{multline*}
    \label{eq:byparts.cyclic.tworounds}
    \Int{e^{x} \sin x}{x} = e^{x} \sin x - \Int{\cos x e^{x}}{x} \\
    = e^{x} \sin x - \pbrac*{\cos x e^{x} - \Int{-\sin x e^{x}}{x}} \\
    = e^{x} \sin x - e^{x} \cos x - \Int{e^{x} \sin x}{x}. \tag{\ddag}
  \end{multline*}
  We're right back where we started!
  This doesn't look good.
  However, there is one spark of light left here: our original integral $\Int{e^{x} \sin x}{x}$ appears on both sides of \cref{eq:byparts.cyclic.tworounds}, so perhaps we can make that work to our advantage.
  If we move both occurrences to the same side of the equation, we end up with
  \begin{equation*}
    2 \Int{e^{x} \sin x}{x} = e^{x} \sin x - e^{x} \cos x,
  \end{equation*}
  which means (after adding in the constant of integration) that
  \begin{equation*}
    \Int{e^{x} \sin x}{x} = \frac{1}{2} e^{x} \sin x - \frac{1}{2} e^{x} \cos x + C.
  \end{equation*}
\end{soln}

Thus, we can use parts even to compute integrals where both factors have cyclic derivatives, as long as we're careful with how we choose our $u$ and $\diff{v}$.
(In \cref{ex:byparts.cyclic}, for example, we had to let $\diff{v} = e^{x} \, \diff{x}$ both times for this to work.)

We can also use integration by parts to find some elementary antiderivatives that are inaccessible otherwise.
For example, in our exploration so far, we've never stumbled on an antiderivative for the natural log function.
With parts, we can find one!

\begin{example}
  \label{ex:byparts.ln}
  Find $\Int{\ln x}{x}$ using integration by parts.
\end{example}

\begin{soln}
  We don't have an antiderivative of $\ln$ available to us, but we do know that its derivative is $\frac{1}{x}$.
  Thus, we'd like to take $u = \ln x$, which forces the (somewhat strange-looking) choice $\diff{v} = 1 \, \diff{x}$; given these choices, we have that $\diff{u} = \frac{1}{x} \diff{x}$ and $v = x$.
  Using these terms in \cref{eq:int.byparts}, we find that
  \begin{equation*}
    \Int{\ln x}{x} = x \ln x - \Int{x \frac{1}{x}}{x} = x \ln x - \Int{1}{x} = x \ln x - 1 + C.
  \end{equation*}
  So letting $\diff{v} = 1 \, \diff{x}$ can pay off!
\end{soln}

\begin{exercises}
\end{exercises}
\end{document}

%%% Local Variables:
%%% mode: latex
%%% TeX-master: "../book/calcnotes.tex"
%%% End: