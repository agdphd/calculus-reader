\documentclass[../book/calcnotes.tex]{subfiles}

\begin{document}
\section{Trigonometric derivatives}
\label{sec:deriv.trig}

Another important class of functions comes from \emph{trigonometry}, the study of the geometry of triangles.
\begin{smallfig}
  \begin{asy}
    smallsize();

    import geometry;
    import math;

    real theta = 30;

    pair a = (0, 0);
    pair b = (1, 0);
    pair c = (1, Sin(theta));

    draw(a--b--c--cycle);

    label("$C$", a--c, LeftSide);
    label("$A$", b--c, RightSide);
    label("$B$", a--b, RightSide);

    markrightangle(a, b, c);
    markangle("$\theta$", b, a, c);
  \end{asy}
  \caption{A triangle for trigonometry}
  \label{fig:trig.triangle}
\end{smallfig}
Given a right triangle like the one in \cref{fig:trig.triangle}, we define six trigonometric functions, corresponding to the six possible ratios of side lengths:
\begin{align*}
  \sin \theta &= \frac{A}{C} & \csc \theta &= \frac{C}{A} = \frac{1}{\sin \theta} \\
  \cos \theta &= \frac{B}{C} & \sec \theta &= \frac{C}{B} = \frac{1}{\cos \theta} \\
  \tan \theta &= \frac{A}{B} = \frac{\sin \theta}{\cos \theta} & \cot \theta &= \frac{B}{A} = \frac{1}{\tan \theta}
\end{align*}

Crucially, it turns out that these ratios depend only on the angle $\theta$ and not on the size of the triangle, so we can think of these as \emph{functions of $\theta$}.
The sine and cosine functions turn out to have particularly nice-looking graphs, as shown in \cref{fig:sin,fig:cos}.

\begin{medfig}
  \begin{asy}
    medsize();

    real f(real x) {return sin(x);}

    real xstart=-1, xend=13;
    real ystart=-1.2, yend=1.4;

    xaxis("$x$", xstart, xend,Arrow);
    yaxis("$y$", ystart, yend, Arrow);

    funcplot(f, xstart, xend);
  \end{asy}
  \caption{Plot of $y = \sin x$}
  \label{fig:sin}
\end{medfig}

\begin{medfig}
  \begin{asy}
    medsize();

    real f(real x) {return cos(x);}

    real xstart=-1, xend=13;
    real ystart=-1.2, yend=1.4;

    xaxis("$x$", xstart, xend,Arrow);
    yaxis("$y$", ystart, yend, Arrow);

    funcplot(f, xstart, xend);
  \end{asy}
  \caption{Plot of $y = \cos x$}
  \label{fig:cos}
\end{medfig}

These functions will turn out to be surprisingly important in calculus.
To see why, let's take a look at their derivatives.

\subsection{Derivative of the sine function}
\label{sec:deriv.sin}
Let's try to calculate $\D{}{x} \sin x$.

Following \cref{def:deriv.func}, we know that
\begin{align*}
  \D{}{x} \sin x
  &= \lim_{\Delta x \to 0} \frac{\sin \pbrac*{x + \Delta x} - \sin \pbrac{x}}{\Delta x}.
  \intertext{
    Trig functions like $\sin$ and $\cos$ have many delightful algebraic properties.
    Here, we'll take advantage of the fact that $\sin \pbrac{a + b} = \sin \pbrac{a} \cos \pbrac{b} + \sin \pbrac{b} \cos \pbrac{a}$ to unpack the numerator of the right-hand side:
  }
  &= \lim_{\Delta x \to 0} \frac{\sin \pbrac*{x} \cos \pbrac*{\Delta x} + \sin \pbrac*{\Delta x} \cos \pbrac{x} - \sin \pbrac{x}}{\Delta x} \\
  &= \lim_{\Delta x \to 0} \frac{\sin \pbrac*{x} \cos \pbrac*{\Delta x} - \sin \pbrac{x}}{\Delta x} + \lim_{\Delta x \to 0} \frac{\sin \pbrac*{\Delta x} \cos \pbrac{x}}{\Delta x} \\
  &= \pbrac*{\lim_{\Delta x \to 0} \frac{\cos \pbrac*{\Delta x} - 1}{\Delta x}} \cdot \sin x
  + \pbrac*{\lim_{\Delta x \to 0} \frac{\sin \pbrac*{\Delta x}}{\Delta x}} \cdot \cos x.
  \intertext{
    These two limits are both quite tricky.
    We'll consider them in more detail in \cref{sec:lhospital}; for now, we'll just note that the first converges to $0$ and the second to $1$.
    With this in hand, we can continue:
  }
  &= 0 \cdot \sin x + 1 \cdot \cos x \\
  &= \cos x.
\end{align*}

So $\D{}{x} \sin x = \cos x$!
This is a remarkably straightforward answer to what might have been a very complicated question.
If we examine \cref{fig:sin,fig:cos} again, though, it actually isn't so surprising.

Specifically, let's consider the interval $\closedint{0, 2\pi}$.
$\sin x$ has positive slopes on $\openint*{0, \frac{\pi}{2}}$ and $\openint*{\frac{3\pi}{2}, 2\pi}$, and $\cos x$ has positive values on these same intervals.
Similarly, $\sin x$ has negative slopes on $\closedint*{\frac{\pi}{2}, \frac{3\pi}{2}}$, and $\cos x$ has negative values on this same interval.

\subsection{Derivatives of the other trig functions}
\label{sec:deriv.trig.other}
What about $\D{}{x} \cos x$?
The argument we just used will work equally well for $\cos$ as for $\sin$.
We just need the cosine addition formula
\begin{equation*}
  \cos \pbrac*{a + b} = \cos a \cos b - \sin a \sin b
\end{equation*}
to get us started.
The details are essentially identical to what we did in \cref{sec:deriv.sin}, so we'll omit the computation here.
The result is just as nice as before: $\D{}{x} \cos x = -\sin x$.
(To see why the negative sign shows up, take another look at \cref{fig:sin,fig:cos}.
We'll wait!)

We could also use similar arguments to compute the derivatives of the remaining four trigonometric functions.
However, it turns out to be easier to use algebraic methods, which we'll develop in \cref{sec:deriv.prodquot}, so we'll leave these aside for now.
To close out the section, let's record the results of our work in a theorem!

\begin{theorem}
  \label{thm:deriv.sincos}
  The derivatives of the sine and cosine functions are given by
  \begin{align}
    \D{}{x} \sin x &= \cos x \label{eq:deriv.sin} \\
    \D{}{x} \cos x &= -\sin x. \label{eq:deriv.cos}
  \end{align}
\end{theorem}

\begin{exercises}
\end{exercises}
\end{document}

%%% Local Variables:
%%% mode: latex
%%% TeX-master: "../book/calcnotes.tex"
%%% End: