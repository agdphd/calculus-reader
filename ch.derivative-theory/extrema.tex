\documentclass[../book/calcnotes.tex]{subfiles}

\begin{document}
\section{Extreme values}
\label{sec:deriv.extrema}

In many applications (such as those in \cref{sec:optimization}), we'll be interested in finding places where a function takes on a maximum or minimum value.
Differential calculus turns out to be a surprisingly powerful tool for addressing this problem.

\subsection{Local extrema}
\label{sec:deriv.extrema.local}
As a first step in this process, we'll define so-called \enquote{local} extrema\footnote{
  \enquote{Extrema} is the plural of \enquote{extreme}, and refers to both maximum and minimum values (the plurals of which are \enquote{maxima} and \enquote{minima} respectively).
}---places where the function takes on its largest or smallest value \emph{in some region}.
Local maxima and minima of a function are illustrated in \cref{fig:local-extrema}.

\begin{smallfig}
  \begin{asy}
    smallsize();

    real f(real x) {return 1 + abs(x)*cos(x);}

    real start=-.75, end=1.5;

    pair a = (0, f(0));
    pair b = (0.86, f(0.86));

    xaxis("$x$", start, end, Arrow);
    yaxis("$y$", 0, Arrow);

    funcplot(f, start, end);

    dot(a);
    dot(b);
    arrow("min", a, SE);
    arrow("max", b, NE);
  \end{asy}
  \caption{A function with a local max and a local min}
  \label{fig:local-extrema}
\end{smallfig}

Let's define this formally so we have a solid place to stand.

\begin{definition}
  \label{def:local-extrema}
  Let $f$ be a function and $p$ be a point in the domain of $f$.
  Then $f$ has a \defterm{local minimum at $p$}[local extrema!minimum] if $f(p) \leq f(x)$ for all values of $x$ sufficiently close to $p$, and $f$ has a \defterm{local maximum at $p$}[local extrema!maximum] if $f(p) \geq f(x)$ for all values of $x$ sufficiently close to $p$.
  If $p$ is either a local minimum or local maximum of $f$, we say $p$ is a \defterm{local extremum of $f$}[local extrema].
\end{definition}

What does this have to do with calculus?
Let's take a closer look at a point where a function has a local extreme, as shown in \cref{fig:local-extrema}.
Between the shown minimum and the maximum, the function $f$ is increasing; before the minimum and after the maximum, it is decreasing.
Right at each extremum, however, $f$ is neither increasing nor decreasing!
At the maximum, the curve $y = f(x)$ has a horizontal tangent line, and at the minimum it has no tangent line at all, as discussed in \cref{gp:deriv.abs}.

In fact, an extremum of $f$ can \emph{only} occur under these circumstances---either $f'(x) = 0$ or $f'(x)$ is undefined at every local extreme $x$.
In essence, this is because an extremum can only occur when the \enquote{increasing-decreasingness} of a function changes---that is, it changes from increasing to decreasing (giving a maximum) or from decreasing to increasing (giving a minimum).
This means that the derivative $f'$ must change signs (from negative to positive or from positive to negative depending on the type of extremum).
Functions can change signs in two ways: by passing through $0$ (in which case $f'(x) = 0$) or by being discontinuous (in which case $f'(x)$ does not exist).

To record this in formal language, we should first give a name to these kinds of points.

\begin{definition}
  \label{def:critical-points}
  Let $f$ be a function and $p$ be a point in the domain of $f$.
  Then $p$ is a \defterm{critical point of $f$}[critical point] if either $f'(p) = 0$ or $f'(p)$ is undefined.
\end{definition}

With this definition in hand, we can state our observation as a theorem.
This theorem is often attributed to French mathematician Pierre de Fermat, whose work helped form the foundation for the development of calculus, so we'll give it his name.

\begin{theorem}[Fermat's theorem for critical points]
  \label{thm:critical-points}
  \index{Fermat's theorem}\index{critical point!Fermat's theorem}
  Let $f$ be a function which has a local extremum at $p$.
  Then $p$ is a critical point of $f$.
\end{theorem}

We should take note here of a very important caveat to this theorem.
Every local extremum of $f$ is found at a critical point---but a function can have critical points which are \emph{not} extrema!
We'll consider this in more detail in \cref{gp:cp-not-extreme}.

What the theorem \emph{does} give us, however, is a powerful tool to \emph{start} looking for extrema of a function.
The domain of a function typically contains infinitely many points, so looking for the extrema \enquote{by hand} would literally take forever!
However, with \cref{thm:critical-points} in hand, we can restrict our search; extrema can \emph{only} occur at critical points, so we can start by finding those and then check each one to see if it's an extremum.

In order to check whether a given critical point actually is an extremum, we need to go a little further.


\subsection{Global extrema}
\label{sec:deriv.extrema.global}

\begin{gps}
    \begin{gp}
    \label{gp:cp-not-extreme}
    Find and classify the critical points of $f(x) = x^{4} - x^{3}$.
    Does this function have any local extrema?
    If so, where are they?
  \end{gp}
\end{gps}

\begin{exercises}
\end{exercises}
\end{document}

%%% Local Variables:
%%% mode: latex
%%% TeX-master: "../book/calcnotes.tex"
%%% End: