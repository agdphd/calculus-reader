\documentclass[../book/calcnotes.tex]{subfiles}

\begin{document}
\section{Antiderivatives}
\label{sec:antiderivatives}

At this point, we've developed a pretty robust theory of differentiation.
In particular, we're pretty good at answering questions of this form:
\begin{quote}
  Given a function $f$, does it have a derivative?
  If so, what is $\D{}{x} f$?
\end{quote}

To a mathematician, this question immediately suggests another, called its \enquote{inverse problem}:
\begin{quote}
  Given a function $f$, are there any functions $F$ such that $\D{}{x} F = f$?
  If so, what are they?
\end{quote}


We've already seen that answers to the first kind of question can be very useful---for example, given the position function $p(t)$ of an object, we can differentiate it to find its velocity function $v(t) = p'(t)$.
In \cref{sec:derivative-applications}, we'll explore an array of other powerful applications.

What about the second kind?
It turns out that finding these \enquote{anti-derivatives} is often just as useful.
For example, when you drive a car, its speedometer measures the car's velocity function $v(t)$.
To find how far you've traveled over a period of time, you'll need to find your position function $p$, using only the fact that $p'(t) = v(t)$.

To start, let's formalize the thing we're looking for with a definition.

\begin{definition}
  \label{def:antiderivative}
  Let $f$ be a function which is continuous on an interval $I = \closedint{a, b}$ and let $F$ be continuous\footnote{In fact, differentiable functions are automatically continuous, but it's worth bearing this condition in mind.} and differentiable on the same interval.
  Then $F$ is an \defterm{antiderivative of $f$ on $I$}[antiderivative] if $\D{F}{x} = f \pbrac{x}$ on $I$.
\end{definition}

Now, to business.
How can we \emph{find} an antiderivative of a function?

\subsection{Antidifferentiation theorems from differentiation theorems}
\label{sec:deriv.antideriv.making-theorems}

We've already developed many theorems telling us how to differentiate functions.
For example, \cref{thm:deriv.exp} tells us that $\D{}{x} b^{x} = \pbrac*{\ln b} b^{x}$ for any base $b > 0$.
We can immediately turn this into an \emph{antidifferentiation} theorem simply by reversing it: $b^{x}$ is an antiderivative of $\pbrac*{\ln b} b^{x}$.
We can do this for any of the derivative theorems we've proved so far!
Thus, we get a library of antidifferentiation theorems \enquote{for free}, just by looking back over the differentiation theorems in the preceding sections.

In practice, when writing antidifferentiation results, we usually prefer to move things around so that the second function is easy to recognize.
For exponential functions, for example, we might instead write \enquote{$\frac{b^{x}}{\ln b}$ is an antiderivative of $b^{x}$}.

For now, we won't bother writing these theorems down, since they're just replications of theorems in previous sections.
Later, we'll develop more sophisticated tools that will allow us to find some more surprising antiderivative theorems.

\subsection{The general antiderivative}
\label{sec:deriv.antideriv.general}

So far, we've spoken exclusively in terms of \enquote{\emph{an} antiderivative} of a given function $f$.
It's important to bear in mind, though, that a given function will generally have \emph{lots} of antiderivatives!
For example, $F(x) = 2x$ is an antiderivative of $f(x) = 2$, and so is $G(x) = 2x + 3$.
Can we say anything general about the set of \emph{all} antiderivatives of a given function?

Suppose $f$ is continuous\footnote{We'll return to the assumption of continuity later. It turns out to be very important!} on an interval $I = \closedint{a, b}$ and that $F$ and $G$ are antiderivatives of $f$ on $I$.
(In other words, $\D{}{x} F(x) = \D{}{x} G(x) = f(x)$.)
What can we say about the relationship between $F$ and $G$?
As a place to start, let's think about their \emph{difference function} $D(x) = F(x) - G(x)$.
Since $F$ and $G$ are both differentiable, $D$ is too, and
\begin{equation*}
  \D{}{x} D(x) = \D{}{x} \pbrac*{F(x) - G(x)} = \D{}{x} F(x) - \D{}{x} G(x) = f(x) - f(x) = 0.
\end{equation*}
So the derivative of $D$ is the zero function!

Recall that one of the conclusions of \cref{thm:deriv.sign} was that any function with zero derivative must be a constant function.
This means that $D$ is constant, so there's some constant $C \in \RR$ such that $F(x) - G(x) = C$ over all of $I$!

This will turn out to be very important, so let's record it in a theorem.

\begin{theorem}
  \label{thm:antiderivative.constant}
  Let $F$ and $G$ be antiderivatives of a continuous function $F$.
  Then there is some constant $C \in \RR$ such that $F(x) = G(x) + C$.
  In other words, if $f$ is a continuous function, all its antiderivatives are equal up to a constant term.
\end{theorem}

If all the antiderivatives of a given function $f$ are equal up to a constant, it's easy to describe the whole set of them at once!
Let's give that a name.

\begin{definition}
  \label{def:antiderivative.general}
  Let $F$ be an antiderivative of a continuous function $f$.
  Then the \defterm{general antiderivative of $f$} is the set of functions $\AD{}{x} f(x) = \setbuilder{F(x) + C}{C \in \RR}$ where $C$ takes values from $\RR$.
  In practice, we write $\AD{}{x} f(x) = F(x) + C$, omitting the set notation.
  % TODO: This definition feels pretty technical. Trim it down?
\end{definition}

\begin{note}
  In many settings, the general antiderivative of $f$ is denoted by $\Int{f(x)}{x}$ instead of $\AD{}{x} f(x)$, for reasons that will become clear in \cref{sec:integral-theory}.
  We'll stick with the $\AD{f}{x}$ notation for now, but you should be aware of this other one in case you see it show up in other settings.
\end{note}

This definition has two extremely important caveats.
First, it requires that we know at least one antiderivative of $f$ from the start.
What if $f$ has no antiderivatives at all?
It will turn out that this isn't much of a problem---in \cref{sec:ftc} we'll see how to construct an antiderivative of \emph{any} continuous function!---but we should definitely bear it in mind.

The second is more subtle.

\subsubsection{Antiderivatives of discontinuous functions}
% TODO: Write this!

\begin{exercises}
\end{exercises}
\end{document}

%%% Local Variables:
%%% mode: latex
%%% TeX-master: "../book/calcnotes.tex"
%%% End: