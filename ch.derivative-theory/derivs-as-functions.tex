\documentclass[../book/calcnotes.tex]{subfiles}

\begin{document}
\section{Derivatives as functions}
\label{sec:derivs-as-functions}

The notion of \emph{derivative}, introduced in \cref{def:deriv}, gives us an analytic tool for studying the slopes of tangent lines to curves.
In particular, given a function $f$ and a value $x = a$, we now have a way to find the slope of the tangent line to the curve $y = f(x)$ at $x = a$.
Of course, the result can depend on the value of $a$ as well as on the function $f$, so in fact the derivative is a \emph{function} which is \enquote{derived from}\footnote{This is, in fact, the origin of the term \enquote{derivative}.} $f$.
We'll record this idea in a definition.

\begin{definition}
  \label{def:deriv.func}
  Fix a function $f$.
  The \emph{derivative function of $f$} (or \emph{derivative of $f$} for short) is the function $f'$ (read \enquote{$f$ prime}) given by
  \begin{equation}
    \label{eq:deriv.func}
    f'(x) = \lim_{\Delta x \to 0} \frac{f \pbrac{x + \Delta x} - f \pbrac{x}}{\Delta x}
  \end{equation}
  wherever the limit exists.
  If $f'$ exists on some region $I$, we say that $f$ is \emph{differentiable on $I$}; if $f'$ exists on the whole domain of $f$, we say that $f$ is \emph{differentiable}.
  For convenience, we will sometimes denote\footnotemark the derivative of $f$ by $\D{f}{x}$ or $\D{}{x} f$ instead of $f'$.
\end{definition}

\footnotetext{
  The \enquote{prime notation} $f'$ comes from Italian mathematician Joseph-Louis Lagrange; the \enquote{differential notation} $\D{f}{x}$ or $\D{}{x} f$ comes from German mathematician Gottfreid Leibniz.
  We will use them both throughout the text.
  Newton used a \enquote{dot notation} $\dot{f}$, but this is now rarely used outside a few scientific settings.
}

If we can find the derivative of a given function $f$, there will be an immediate payoff: we'll have calculated the slopes of \emph{all} the tangent lines to $y = f(x)$ at once!
In many cases, we'll need special techniques and careful analysis to find the derivatives of functions (which will occupy us for the rest of \cref{sec:derivative-theory}), but there are some functions we can tackle directly.

\begin{example}
  \label{ex:derivative.quadratic.function}
  Find the derivative of $f(x) = x^{2}$.
  Use the result to find the tangent line to $y = x^{2}$ at $x = 1$, $x = 2$, and $x = -3$.
\end{example}

\begin{soln}
  Recall that we already computed $f'(1)$ in \cref{ex:derivative.quadratic.onepoint}.
  To find $f'(x)$ as a function of $x$, we only need to retrace our steps, using $x$ instead of the specific value $x = 1$.
  By \cref{def:deriv.func}, the derivative is
  \begin{multline*}
    f'(x) = \lim_{\Delta x \to 0} \frac{\pbrac*{x + \Delta x}^{2} - \pbrac{x}^{2}}{\Delta x} \\
    = \lim_{\Delta x \to 0} \frac{x^{2} + 2 x \Delta x + \pbrac*{\Delta x}^{2} - x^{2}}{\Delta x} = \lim_{\Delta x \to 0} \frac{2 x \Delta x + \pbrac*{\Delta x}^{2}}{\Delta x} \\
    = \lim_{\Delta x \to 0} 2 x + \Delta x = 2x.
  \end{multline*}
  In differential notation, this result is
  \begin{equation*}
    \D{}{x} x^{2} = 2x.
  \end{equation*}
  We can translate this directly into tangent line information.
  At $x = 1$, the tangent line has slope $f'(1) = 2$; at $x = 2$, the tangent line has slope $f'(2) = 4$; and at $x = 3$, the tangent line has slope $f'(-3) = -6$.
  By \cref{thm:derivtan}, the tangent lines then have the following equations:
  \begin{gather*}
    y = f(1) + f'(1) \pbrac{x - 1} = 1 + 2 \pbrac{x - 1} \tag{x = 1} \\
    y = f(2) + f'(2) \pbrac{x - 2} = 4 + 4 \pbrac{x - 2} \tag{x = 2} \\
    y = f(-3) + f'(-3) \pbrac{x - -3} = 9 - 6 \pbrac{x + 3} \tag{x = -3}
  \end{gather*}
\end{soln}

This is our first application of the idea of a derivative function: for a given function $f$, the derivative function $f'$ tells us the slopes of \emph{all} the tangent lines to $y = f(x)$ at once!

\subsection{Rates of change}
\label{sec:deriv.roc}

All of this is great if, like the analytic geometers of sixteenth-century Europe, we want to draw tangent lines to curves.
However, we should take a moment to consider whether the concept of the derivative is of any other use.

Suppose, for example, that an object is moving back and forth in a straight line such that its position at time $t$ (in \si{\second}) is $x(t)$ (in \si{\meter}).
What does the number $x' \pbrac{1}$ measure?
Of course, we've already given one interpretation: it's the slope of the tangent line to\footnote{Note that this is a line in the $y$-$t$ plane! $x$ is the name of a \emph{function} here.} $y = x(t)$ at $t = 1$.
But does it tell us anything about the motion of the object itself?

In light of \cref{def:deriv}, we know that
\begin{equation*}
  x'(1) = \lim_{\Delta t \to 0} \frac{x \pbrac*{1 + \Delta t} - x \pbrac{1}}{\Delta t}.
\end{equation*}
Let's disregard the limit for a moment and consider the numbers $\frac{x \pbrac*{1 + \Delta t} - x \pbrac{1}}{\Delta t}$ for small (but non-zero!) values of $\Delta t$.
When $\Delta t = \SI{0.5}{\second}$, for example, this number is $\frac{x \pbrac*{\SI{1.5}{\second}} - x \pbrac*{\SI{1}{\second}}}{\SI{0.5}{\second}}$.
Without knowing more about the function $x$, of course, we can't calculate what number this is, but we can certainly say something about what it \emph{measures}: it's the average velocity of the object (in \si{\meter\per\second}) over the time interval $t \in \closedint{\SI{1}{\second}, \SI{1.5}{\second}}$!
Indeed, for any nonzero $\Delta t$ we choose, the number $\frac{x \pbrac*{1 + \Delta t} - x \pbrac{1}}{\Delta t}$ measures the average velocity of the object over the interval $t \in \closedint{\SI{1}{\second}, \pbrac*{1 + \Delta t} \si{\second}}$.
There's nothing special about the number $1$, of course, so this leads us to a general result: \emph{the derivative of a position function is a velocity function.}

In fact, we can generalize this interpretation far beyond mechanical motion.
Any time we are interested in the values of a function $f$, the values of the derivative function $f'$ measure the \emph{rate of change} of the values of $f$ with respect to the independent variable.
Here's just a few examples:

\begin{itemize}
\item
  If a cup of coffee set on a counter has temperature $T$ (in \si{\celsius}) at time $t$ (in \si{\minute}), the derivative $T'(t)$ measures the rate at which the coffee cools (in \si{\celsius\per\minute}).

\item
  If your browser has downloaded $S$ bytes of a file after $t$ seconds, the derivative $S'(t)$ measures the download throughput (in \si{\byte\per\second}).

\item
  If the cost of manufacturing $n$ computers is $C(n)$ U.S.~dollars, then $C'(n)$ is the \emph{marginal cost}, which measures the cost of producing a single computer given that $n$ have already been produced.
  (Notice that the independent variable doesn't have to be time!)

\item
  The pressure $P$ (in \si{\pascal}) on a diver is a function of the depth $d$ (in \si{\meter}) of the dive.
  The derivative $P'(d)$ measures how quickly the pressure increases as the diver descends.
\end{itemize}

This idea connects calculus to essentially every quantitative science, because it implies that differentiation is relevant to the study of anything that changes!
We'll revisit this idea down the road once we know how to find the derivatives of more functions.

\subsection{Properties of derivatives}
\label{sec:derivprops}

\begin{exercises}
\end{exercises}
\end{document}

%%% Local Variables:
%%% mode: latex
%%% TeX-master: "../book/calcnotes.tex"
%%% End: