\documentclass[../book/calcnotes.tex]{subfiles}

\begin{document}
\section{Properties of the derivative}
\label{sec:deriv.properties}

The concept of the \emph{derivative}, established in \cref{def:deriv.func}, will prove to be an extremely powerful tool for analyzing functions in many settings in mathematics and the sciences.
As a first step toward studying it, let's consider an important basic class of functions.

\subsection{Derivatives of power functions}
\label{sec:deriv.polynomials}

Back in \cref{ex:derivative.quadratic.function}, we showed that $\D{}{x} x^{2} = 2x$.
What happens if we use a power other than $2$?

\begin{example}
  \label{ex:derivative.cubic.function}
  Find the derivative of $f(x) = x^{3}$.
\end{example}

\begin{soln}
  By \cref{def:deriv.func}, we compute
  \begin{align*}
    f'(x)
    &= \lim_{\Delta x \to 0} \frac{\pbrac*{x + \Delta x}^{3} - \pbrac*{x}^{3}}{\Delta x} \\
    &= \lim_{\Delta x \to 0} \frac{x^{3} + 3 x^{2} \Delta x + 3 x \pbrac*{\Delta x}^{2} + \pbrac*{\Delta x}^{3} - x^{3}}{\Delta x} \\
    &= \lim_{\Delta x \to 0} \frac{3 x^{2} \Delta x + 3x \pbrac*{\Delta x}^{2} + \pbrac*{\Delta x}^{3}}{\Delta x} \\
    &= \lim_{\Delta x \to 0} 3x^{2} + 3 x \Delta x + \pbrac*{\Delta x}^{2} = 3x^{2}.
  \end{align*}
  So $\D{}{x} x^{3} = 3x^{2}$.
\end{soln}

Of course, we could keep going---ideally, we'd like to compute the derivative of $f(x) = x^{n}$ for any $n$---but the algebra in the numerator is going to start getting pretty messy.
Fortunately, mathematicians of yore have done the heavy lifting for us by proving the following result:
\begin{lemma}
  \label{thm:diffofpowers}
  Let $a$ and $b$ be real numbers and $n$ be a positive integer.
  Then
  \begin{equation}
    \label{eq:diffofpowers}
    a^{n} - b^{n} = \pbrac*{a - b} \pbrac*{a^{n-1} + a^{n-2} b + a^{n-3} b^{2} + \dots + a^{2} b^{n-3} + a b^{n-2} + b^{n-1}}.
  \end{equation}
\end{lemma}

If we let $a = x + \Delta x$ and $b = x$, the left-hand side of \cref{eq:diffofpowers} looks very much like the numerator of the difference quotient that showed up in the calculation of $\D{}{x} x^{3}$.
What if we try to take the derivative of a general power function $f(x) = x^{n}$?
Again, following \cref{def:deriv.func}, we'll compute
\begin{align*}
  \D{}{x} x^{n}
  &= \lim_{\Delta x \to 0} \frac{\pbrac*{x + \Delta x}^{n} - \pbrac*{x}^{n}}{\Delta x}. \\
  \intertext{If we apply \cref{thm:diffofpowers} to the numerator, we can expand it, so}
  &= \lim_{\Delta x \to 0} \frac{\pbrac*{x + \Delta x - x} \pbrac*{\pbrac*{x + \Delta x}^{n-1} + \pbrac{x + \Delta x}^{n-2} x + \dots + \pbrac*{x + \Delta x} x^{n-2} + x^{n-1}}}{\Delta x} \\
  &= \lim_{\Delta x \to 0} \frac{\Delta x \cdot \pbrac*{\pbrac*{x + \Delta x}^{n-1} + \pbrac{x + \Delta x}^{n-2} x + \dots + \pbrac*{x + \Delta x} x^{n-2} + x^{n-1}}}{\Delta x} \\
  &= \lim_{\Delta x \to 0} \pbrac*{x + \Delta x}^{n-1} + \pbrac{x + \Delta x}^{n-2} x + \dots + \pbrac*{x + \Delta x} x^{n-2} + x^{n-1}.
\end{align*}
Now we can take the limit by simply setting $\Delta x = 0$, so we conclude that
\begin{equation*}
  \D{}{x} x^{n} = x^{n-1} + x^{n-2} \cdot x + \dots + x \cdot x^{n-2} + x^{n-1} = n x^{n-1}.
\end{equation*}

Of course, this calculation only makes sense if $n$ is a positive integer.
Ultimately, we'd like to be able to differentiate \emph{any} power function, but it turns out that we'll need more powerful machinery to do it.
Remarkably, it turns out that the same result holds!
We'll go ahead and record it here, then revisit the proof later when we have the tools to do it.

\begin{theorem}[Derivatives of power functions]
  \label{thm:deriv.power}
  Let $c$ be a real number.
  Then $f(x) = x^{c}$ is differentiable, and
  \begin{equation}
    \label{eq:deriv.power}
    \D{}{x} x^{c} = c x^{c-1}.
  \end{equation}
\end{theorem}

\subsection{Algebra of derivatives}
\label{sec:deriv.algebra}
So far, we have two tools to compute derivatives: \cref{def:deriv.func,thm:deriv.power}.
We'll calculate many more derivatives of specific functions in the coming sections.
Before we do, though, let's take a moment to think about the \emph{algebra} of differentiation.

\subsubsection{Constant multiples}
What happens, for example, if we take a derivative of a constant multiple of a function?
Let's think about this in terms of the geometry of tangent lines.
Suppose $f$ is a differentiable function.
The graphs of $y = f(x)$ and $y = 3f(x)$ are similar; the second is obtained from the first by \enquote{stretching} it vertically by a factor of $3$.
This will cause all the tangent lines to become three times steeper, suggesting that $\D{}{x} 3f(x) = 3 \D{}{x} f(x)$.

Will this kind of reasoning always work?
To check it formally, let's go to the algebra of \cref{def:deriv.func}.

\begin{example}
  \label{ex:deriv.constmult.alg}
  Let $f$ be a differentiable function.
  Compare $\D{}{x} 3f(x)$ to $\D{}{x} f(x)$ using \cref{def:deriv.func}.
\end{example}

\begin{soln}
  In light of \cref{def:deriv.func}, we compute
  \begin{align*}
    \D{}{x} 3 f(x)
    &= \lim_{\Delta x \to 0} \frac{\pbrac{3f} \pbrac{x + \Delta x} - \pbrac{3f} \pbrac{x}}{\Delta x} \\
    &= \lim_{\Delta x \to 0} \frac{3 \pbrac*{f \pbrac*{x + \Delta x}} - 3 \pbrac*{f \pbrac{x}}}{\Delta x} \\
    &= \lim_{\Delta x \to 0} \frac{3 \pbrac*{f \pbrac*{x + \Delta x} - f \pbrac*{x}}}{\Delta x} \\
    &= 3 \lim_{\Delta x \to 0} \frac{f \pbrac*{x + \Delta x} - f \pbrac*{x}}{\Delta x} \\
    &= 3 \D{}{x} f(x).
  \end{align*}
  So $\D{}{x} 3 f(x) = 3 \D{}{x} f(x)$.
\end{soln}

And there we have it!
In fact, this exact same argument will work if we replace $3$ with any other constant real number.
Thus, we get our first algebraic result about derivatives.

\begin{theorem}[Constant multiple rule for derivatives]
  \label{thm:deriv.constmult}
  Let $f$ be a differentiable function and $k$ be a real number.
  Then
  \begin{equation}
    \label{eq:deriv.constmult}
    \D{}{x} k f(x) = k \D{}{x} f(x) = k f'(x).
  \end{equation}
\end{theorem}

\begin{example}
  \label{ex:deriv.cubic.bythm}
  Find $\D{}{x} 7 x^{12}$.
\end{example}

\begin{soln}
  By \cref{thm:deriv.power}, $\D{}{x} x^{12} = 12 x^{11}$.
  Thus, by \cref{thm:deriv.constmult}, $\D{}{x} 7 x^{12} = 7 \D{}{x} x^{12} = 7 \cdot 12 x^{11} = 84 x^{11}$.
\end{soln}

\subsubsection{Addition of functions}
Now suppose we want to compute the derivative of $f + g$ for differentiable functions $f$ and $g$.
This time, let's think in terms of rates of change.

Suppose a train is rolling through the countryside with position $T$ (in \si{\meter}) at time $t$ (in \si{\minute}) relative to the station.
Meanwhile, a conductor is walking down the aisle of the train, with position $C$ (in \si{\meter}) at time $t$ (in \si{\minute}) relative to the rear of the train.
How fast is the conductor moving relative to the ground?
Clearly her total velocity is equal to the sum of her speed down the aisle and the train's speed down the track, which is given by $C'(t) + T'(t)$.
But this velocity is also the rate of change of her \emph{position} relative to the station, which is given by $\D{}{t} \pbrac*{C(t) + T(t)}$.

Will this kind of reasoning always work?
Once again, we can check using the algebra of \cref{def:deriv.func}.

\begin{example}
  \label{ex:deriv.sum.alg}
  Let $f$ and $g$ be differentiable functions.
  Compare $\D{}{x} f(x) + g(x)$ to $\D{}{x} f(x)$ and $\D{}{x} g(x)$ using \cref{def:deriv.func}.
\end{example}

\begin{soln}
  In light of \cref{def:deriv.func}, we compute
  \begin{align*}
    \D{}{x} f(x) + g(x)
    &= \lim_{\Delta x \to 0} \frac{\pbrac{f + g} \pbrac*{x + \Delta x} - \pbrac{f + g} \pbrac*{x}}{\Delta x} \\
    &= \lim_{\Delta x \to 0} \frac{f \pbrac*{x + \Delta x} + g \pbrac*{x + \Delta x} - f \pbrac{x} - g \pbrac{x}}{\Delta x} \\
    &= \lim_{\Delta x \to 0} \frac{f \pbrac*{x + \Delta x} - f \pbrac{x}}{\Delta x} + \frac{g \pbrac*{x + \Delta x} - g \pbrac{x}}{\Delta x} = f'(x) + g'(x).
  \end{align*}
  Thus, $\D{}{x} f(x) + g(x) = \D{}{x} f(x) + \D{}{x} g(x)$.
\end{soln}

And, once again, there we have it!
We should record this result for future reference.

\begin{theorem}[Sum rule for derivatives]
  \label{thm:deriv.sum}
  Let $f$ and $g$ be differentiable functions.
  Then
  \begin{equation}
    \label{eq:deriv.sum}
    \D{}{x} f(x) + g(x) = \pbrac*{\D{}{x} f(x)} + \pbrac*{\D{}{x} g(x)} = f'(x) + g'(x).
  \end{equation}
\end{theorem}

Taken together, \cref{thm:deriv.constmult,thm:deriv.sum} are sometimes called the \enquote{linearity theorem} for differentiation.

\subsubsection{Constant functions}
Now suppose we want to compute the derivative of a constant function (that is, a function $f$ such that $f(x) = k$ for all $x$).
This is easy enough to do directly, using \cref{def:deriv.func}:
\begin{equation*}
  \D{}{x} k = \lim_{\Delta x \to 0} \frac{k - k}{\Delta x} = \lim_{\Delta x \to 0} 0 = 0.
\end{equation*}

We can also interpret this result in terms of rates of change.
Suppose, for example, that the position of an object is constant---in other words, that the object never moves.
Clearly, its velocity is zero!

Once again, we should record this fact future reference.
\begin{theorem}[Constant derivative theorem]
  \label{thm:deriv.const}
  Let $f$ be a function which is constant on $\closedint{a, b}$.
  (In other words, $f(x) = k$ for all $x \in \closedint{a, b}$ for some real number $k$.)
  Then $\D{}{x} f(x) = 0$ on $\closedint{a, b}$.
\end{theorem}

If we take \cref{thm:deriv.sum,thm:deriv.constmult,thm:deriv.const} together, we can compute many derivatives.
For example, if we combine them with \cref{thm:deriv.power}, we can now find the derivative of any polynomial function!

\begin{example}
  \label{ex:deriv.polynomial}
  Let $f(x) = 3x^{4} - x^{3} + 12x^{2} - 4x + 3$.
  Find $f'(x)$.
\end{example}

\begin{soln}
  By \cref{thm:deriv.sum}, we can break up our derivative calculation over the $+$ signs:
  \begin{multline*}
    \D{}{x} \pbrac*{3x^{4} - x^{3} + 12x^{2} - 4x + 3} \\
    = \pbrac*{\D{}{x} 3x^{4}} + \pbrac*{\D{}{x} -x^{3}} + \pbrac*{\D{}{x} 12x^{2}} + \pbrac*{\D{}{x} -4x} + \pbrac*{\D{}{x} 3}.
  \end{multline*}
  \Cref{thm:deriv.constmult} tells us that we can \enquote{pull out} the constant factors from each of these derivatives:
  \begin{multline*}
    \D{}{x} \pbrac*{3x^{4} - x^{3} + 12x^{2} - 4x + 3} \\
    = 3 \pbrac*{\D{}{x} x^{4}} + \pbrac{-1} \pbrac*{\D{}{x} x^{3}} + 12 \pbrac*{\D{}{x} x^{2}} + \pbrac*{-4} \pbrac*{\D{}{x} x} + \pbrac*{\D{}{x} 3}.
  \end{multline*}
  Finally, we can use \cref{thm:deriv.power} and \cref{thm:deriv.const} to compute these derivatives.
  \begin{align*}
    \D{}{x} \pbrac*{3x^{4} - x^{3} + 12x^{2} - 4x + 3}
    &= 3 \cdot 4 x^{3} - 3 x^{2} + 12 \cdot 2 x^{1} - 4 x^{0} + 0 \\
    &= 12 x^{3} - 3x^{2} + 24 x - 4.
  \end{align*}
\end{soln}

\begin{exercises}
\end{exercises}
\end{document}

%%% Local Variables:
%%% mode: latex
%%% TeX-master: "../book/calcnotes.tex"
%%% End: