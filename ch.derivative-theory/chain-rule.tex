\documentclass[../book/calcnotes.tex]{subfiles}

\begin{document}
\section{The chain rule}
\label{sec:chainrule}

So far, we've developed tools to differentiate many classes of functions: constant, power, exponential, trigonometric functions, along with sums and products of these.
However, for many of the applications we'll study in \cref{sec:derivative-applications}, we'll need to differentiate more complicated functions.
For example, we might be interested in the differential properties of the straightforward-looking function $f(x) = \sin 2x$.
None of the tools introduced so far will allow us to differentiate this function!

However, if we think things through carefully, we should be able to make some progress on this problem.
Consider $f(x) = \sin 2x$ again and suppose we want to compute $f' \pbrac*{\sfrac{\pi}{3}}$.
By definition, this is the slope of the tangent line to the curve $y = f(x)$ at $x = \sfrac{\pi}{3}$.
The plot of $y = f(x)$ and the tangent line to it at $x = \sfrac{\pi}{3}$ are illustrated in \cref{fig:chainrule1}.

\begin{smallfig}
  \begin{asy}
    smallsize();

    real xstart=-.2, xend=5.0;
    real ystart=-1.2, yend=2.0;

    real a=1/3*pi;

    xaxis("$x$", xstart, xend, Ticks(new real[] {a}, ticklabel=OmitFormat(a)), Arrow);
    labelx("$a$", a);
    yaxis("$y$", ystart, yend, Arrow);

    real f(real x) {return sin(2*x);}
    real df(real x) {return 2*cos(2*x);}

    path fplot = funcplot(f, xstart, xend/2);
    draw(fplot);

    tanplot(f, df, a, xstart, xend/2, dashed);
  \end{asy}
  \caption{Plot of $y = \sin 2x$ with tangent}
  \label{fig:chainrule1}
\end{smallfig}

Now consider the function $g(x) = \sin x$.
This is clearly related to $f(x)$: specifically, $f(x) = g(2x)$.
Thus, for example, $f \pbrac*{\sfrac{\pi}{3}} = g \pbrac*{\sfrac{2 \pi}{3}}$.
The plot of $y = g(x)$ and the tangent line to it at $x = \sfrac{2 \pi}{3}$ are shown in \cref{fig:chainrule2}.

\begin{smallfig}
  \begin{asy}
    smallsize();

    real xstart=-.2, xend=5.0;
    real ystart=-1.2, yend=2.0;

    real a=2/3*pi;

    xaxis("$x$", xstart, xend, Ticks(new real[] {a}, ticklabel=OmitFormat(a)), Arrow);
    labelx("$a$", a);
    yaxis("$y$", ystart, yend, Arrow);

    real f(real x) {return sin(x);}
    real df(real x) {return cos(x);}

    path fplot = funcplot(f, xstart, xend);
    draw(fplot);

    tanplot(f, df, a, xstart, xend, dashed);
  \end{asy}
  \caption{Plot of $y = \sin x$ with tangent}
  \label{fig:chainrule2}
\end{smallfig}

Naturally, the plot of $g$ and the plot of $f$ are related: the plot of $y = f(x)$ is obtained by \enquote{compressing} the plot of $y = g(x)$ by a factor of $2$ in the horizontal direction.
The tangent lines shown in \cref{fig:chainrule1,fig:chainrule2} are related as well; they are almost the same, but the one in \cref{fig:chainrule1} is twice as steep due to this compression; in other words, $f' \pbrac*{\sfrac{\pi}{3}} = 2 g' \pbrac*{2 \sfrac{\pi}{3}}$.
The re-emergence of the number $2$ here is not a coincidence!
It's showing up because $f$ changes \enquote{twice as fast} as $g$.

Of course, there was nothing special about the point $x = \sfrac{\pi}{3}$.
This same argument will serve to show that $f'(x) = 2 g'(2x)$ for any value $x$ we choose.
Likewise, nothing was special about the function $\sin$ or the multiplier $2$; we can in fact draw a very general conclusion from this line of reasoning.

\begin{theorem}
  \label{thm:chainrule.constmult}
  Fix a real number $k$ and let $g$ be a function which is differentiable at $x = ka$.
  Then the \enquote{compressed function} $f(x) = g(kx)$ is differentiable at $x = a$, and its derivative is given by
  \begin{equation}
    \label{eq:chainrule.constmult}
    f'(x) = \D{}{x} g(kx) = k g'(kx).
  \end{equation}
\end{theorem}

What about composite functions where the inner function is more complicated than this?
It turns out that the situation is still quite similar.
Let $f$ and $g$ be differentiable functions and consider their composite $h(x) = f \pbrac*{g \pbrac{x}}$.
The plot of $y = h(x)$ in the vicinity of some point $x = a$ looks very much like the plot of $y = f(x)$ near the point $x = g(a)$.
Indeed, just as before, the main difference between the two is that $y = h(x) = f(g(x))$ may be \enquote{compressed} or \enquote{stretched}, because the input value $g(x)$ may be changing more or less quickly than $x$ itself.

How quickly is $g(x)$ changing at $x = a$?
Exactly $g'(a)$, of course!
Thus, this situation is really no different than the one we investigated before---the derivative of the composite function $h(x) = f(g(x))$ is proportional to that of the outer function $f$, and the constant of proportionality is given by the rate of change $g'(x)$ of the inner function $g$.
In light of this, we can generalize \cref{thm:chainrule.constmult} to allow us to compute derivatives of arbitrary composite functions.

\begin{theorem}[Chain rule for derivatives]
  \label{thm:chainrule}
  Let $f$ and $g$ be functions such that $g$ is differentiable at $x$ and $f$ is differentiable at $g(x)$.
  Then $f \circ g$ is differentiable at $x$, and
  \begin{equation}
    \label{eq:chainrule}
    \D{}{x} f \pbrac*{g \pbrac*{x}} = g' \pbrac{x} f' \pbrac*{g \pbrac{x}}.
  \end{equation}
\end{theorem}

% TODO: Write out some examples

\subsubsection{A formal proof}
We can prove \cref{thm:chainrule} in a huge variety of ways.
In \cref{sec:productrule}, we proved the Product Rule for derivatives by writing out the definition of the derivative (\cref{eq:deriv.func} from \cref{def:deriv.func}) and working things out computationally.
This same method would work here, but the algebra is even more messy.
Instead, we'll apply what's known as the \enquote{method of increments} to keep things manageable.

Suppose that $y = f(u)$ and $u = g(x)$ for differentiable functions $f$ and $g$.
Let $x_{0}$ be some value of $x$, $u_{0} = g \pbrac*{x_{0}}$ be the corresponding value of $u$, and $y_{0} = f \pbrac*{u_{0}} = f \pbrac*{g \pbrac*{x_{0}}}$ be the corresponding value of $f$.
If $x$ is changed from $x_{0}$ by a small amount $\Delta x$, then $u$ changes from $u_{0}$ by some small amount $\Delta u$ and $y$ changes from $y_{0}$ by some small amount $\Delta y$.
Whatever these numbers are, it is clear that
\begin{equation}
  \label{eq:chainrule.increments}
  \diffquot{y}{x} = \diffquot{y}{u} \cdot \diffquot{u}{x}
\end{equation}
simply by algebra (as long as\footnote{In the case that some of the difference terms are zero, the proof is simpler but less instructive.} $\Delta u$ is nonzero), since these are actual fractions whose numerators and denominators are simply real numbers.

Ultimately, we want to find $\D{y}{x}$.
By the definition of the derivative, this is the limit as $\Delta x$ goes to zero of the term $\diffquot{y}{x}$ on the left-hand side of \cref{eq:chainrule.increments}.
Likewise, on the right-hand side, $\lim_{\Delta x \to 0} \diffquot{u}{x} = \D{u}{x}$.

What about the limit $\lim_{\Delta x \to 0} \diffquot{y}{u}$?
If $\Delta x \to 0$, clearly $\Delta u$ does also, since otherwise $\lim_{\Delta x \to 0} \diffquot{u}{x}$ would be infinite and so $g$ would not be differentiable at $x_{0}$.
Thus, $\lim_{\Delta x \to 0} \diffquot{y}{u} = \lim_{\Delta u \to 0} \diffquot{y}{u} = \D{y}{u}$.

Combining these results together, if we take the limits of both sides of \cref{eq:chainrule.increments} as $\Delta x$ goes to $0$, we conclude that
\begin{equation}
  \label{eq:chainrule.leibniz}
  \D{y}{x} = \D{y}{u} \cdot \D{u}{x}.
\end{equation}

This form of the chain rule was known to Leibniz, who was in general very fond of the so-called \enquote{differential notation} $\D{y}{x}$ for derivatives.

% TODO: Write some exercises!

\begin{exercises}
  \begin{exc}
    \label{exc:deriv.trig.degrees}
    Do calculus to the sine and cosine functions with degrees.
    % TODO: Write this exercise.
  \end{exc}
\end{exercises}
\end{document}

%%% Local Variables:
%%% mode: latex
%%% TeX-master: "../book/calcnotes.tex"
%%% End: