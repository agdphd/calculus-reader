\documentclass[../book/calcnotes.tex]{subfiles}

\begin{document}
\section*{Preface}
\label{sec:deriv.preface}
We'll begin by considering a simple-sounding problem.

\begin{motprob}
  Given a curve $y = f(x)$ and a point $\pbrac{x, y}$ on that curve, what is the tangent line to the curve at the point?
\end{motprob}

This problem has a very long history; it was considered in various forms by mathematicians as far back as Euclid and Archimedes in the third century BCE.
The analytic geometers of seventeenth-century Europe were particularly fascinated by this problem; Descartes, Barrow, and Fermat all invested significant time in it.
Ultimately, it was the development of differential calculus by Newton and Leibniz which allowed the solution of this problem.

Indeed, in Newton's \enquote{Tract on Fluxions} of October 1666, in which he develops many of the methods which would become the modern differential calculus, he proposes as his first problem (of thirteen):
\begin{quote}
  \scripttext{Prob.~1.\quad To draw Tangents to crooked lines.}
\end{quote}

We won't follow all of Newton's methods here---the ideas have become much clearer over the intervening centuries---but we'll take our cue from this question.
\end{document}

%%% Local Variables:
%%% mode: latex
%%% TeX-master: "../book/calcnotes.tex"
%%% End: